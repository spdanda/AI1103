\documentclass[journal,12pt,twocolumn]{IEEEtran}

\usepackage{setspace}
\usepackage{gensymb}
\singlespacing
\usepackage[cmex10]{amsmath}
\usepackage{amsmath}

\usepackage{amsthm}

\usepackage{mathrsfs}
\usepackage{txfonts}
\usepackage{stfloats}
\usepackage{bm}
\usepackage{cite}
\usepackage{cases}
\usepackage{subfig}

\usepackage{longtable}
\usepackage{multirow}
\usepackage{caption}

\usepackage{enumitem}
\usepackage{mathtools}
\usepackage{steinmetz}
\usepackage{tikz}
\usepackage{circuitikz}
\usepackage{verbatim}
\usepackage{tfrupee}
\usepackage[breaklinks=true]{hyperref}
\usepackage{graphicx}
\usepackage{tkz-euclide}
\usepackage{float}
\usepackage{mathtools}

\usetikzlibrary{calc,math}
\usepackage{listings}
    \usepackage{color}                                            %%
    \usepackage{array}                                            %%
    \usepackage{longtable}                                        %%
    \usepackage{calc}                                             %%
    \usepackage{multirow}                                         %%
    \usepackage{hhline}                                           %%
    \usepackage{ifthen}                                           %%
    \usepackage{lscape}     
\usepackage{multicol}
\usepackage{chngcntr}

\DeclareMathOperator*{\Res}{Res}
\newcommand*{\permcomb}[4][0mu]{{{}^{#3}\mkern#1#2_{#4}}}
\newcommand*{\perm}[1][-3mu]{\permcomb[#1]{P}}
\newcommand*{\comb}[1][-1mu]{\permcomb[#1]{C}}
\renewcommand\thesection{\arabic{section}}
\renewcommand\thesubsection{\thesection.\arabic{subsection}}
\renewcommand\thesubsubsection{\thesubsection.\arabic{subsubsection}}

\renewcommand\thesectiondis{\arabic{section}}
\renewcommand\thesubsectiondis{\thesectiondis.\arabic{subsection}}
\renewcommand\thesubsubsectiondis{\thesubsectiondis.\arabic{subsubsection}}


\hyphenation{op-tical net-works semi-conduc-tor}
\def\inputGnumericTable{}                                 %%

\lstset{
%language=C,
frame=single, 
breaklines=true,
columns=fullflexible
}
\begin{document}

\newcommand{\BEQA}{\begin{eqnarray}}
\newcommand{\EEQA}{\end{eqnarray}}
\newcommand{\define}{\stackrel{\triangle}{=}}
\bibliographystyle{IEEEtran}
\raggedbottom
\setlength{\parindent}{0pt}
\providecommand{\mbf}{\mathbf}
\providecommand{\pr}[1]{\ensuremath{\Pr\left(#1\right)}}
\providecommand{\qfunc}[1]{\ensuremath{Q\left(#1\right)}}
\providecommand{\sbrak}[1]{\ensuremath{{}\left[#1\right]}}
\providecommand{\lsbrak}[1]{\ensuremath{{}\left[#1\right.}}
\providecommand{\rsbrak}[1]{\ensuremath{{}\left.#1\right]}}
\providecommand{\brak}[1]{\ensuremath{\left(#1\right)}}
\providecommand{\lbrak}[1]{\ensuremath{\left(#1\right.}}
\providecommand{\rbrak}[1]{\ensuremath{\left.#1\right)}}
\providecommand{\cbrak}[1]{\ensuremath{\left\{#1\right\}}}
\providecommand{\lcbrak}[1]{\ensuremath{\left\{#1\right.}}
\providecommand{\rcbrak}[1]{\ensuremath{\left.#1\right\}}}
\theoremstyle{remark}
\newtheorem{rem}{Remark}
\newcommand{\sgn}{\mathop{\mathrm{sgn}}}
\providecommand{\abs}[1]{\vert#1\vert}
\providecommand{\res}[1]{\Res\displaylimits_{#1}} 
\providecommand{\norm}[1]{\lVert#1\rVert}
%\providecommand{\norm}[1]{\lVert#1\rVert}
\providecommand{\mtx}[1]{\mathbf{#1}}
\providecommand{\mean}[1]{E[ #1 ]}
\providecommand{\fourier}{\overset{\mathcal{F}}{ \rightleftharpoons}}
%\providecommand{\hilbert}{\overset{\mathcal{H}}{ \rightleftharpoons}}
\providecommand{\system}{\overset{\mathcal{H}}{ \longleftrightarrow}}
	%\newcommand{\solution}[2]{\textbf{Solution:}{#1}}
\newcommand{\solution}{\noindent \textbf{Solution: }}
\newcommand{\cosec}{\,\text{cosec}\,}
\providecommand{\dec}[2]{\ensuremath{\overset{#1}{\underset{#2}{\gtrless}}}}
\newcommand{\myvec}[1]{\ensuremath{\begin{pmatrix}#1\end{pmatrix}}}
\newcommand{\mydet}[1]{\ensuremath{\begin{vmatrix}#1\end{vmatrix}}}
\numberwithin{equation}{subsection}
\makeatletter
\@addtoreset{figure}{problem}
\makeatother
\let\StandardTheFigure\thefigure
\let\vec\mathbf
\renewcommand{\thefigure}{\theproblem}
\def\putbox#1#2#3{\makebox[0in][l]{\makebox[#1][l]{}\raisebox{\baselineskip}[0in][0in]{\raisebox{#2}[0in][0in]{#3}}}}
     \def\rightbox#1{\makebox[0in][r]{#1}}
     \def\centbox#1{\makebox[0in]{#1}}
     \def\topbox#1{\raisebox{-\baselineskip}[0in][0in]{#1}}
     \def\midbox#1{\raisebox{-0.5\baselineskip}[0in][0in]{#1}}
\vspace{3cm}
\title{Result from Challenging Problem 1}
\author{Name: Sai Pravallika Danda, Roll Number: CS20BTECH11013}
\maketitle
\newpage
\bigskip
\renewcommand{\thefigure}{\theenumi}
\renewcommand{\thetable}{\theenumi}
Download all latex-tikz codes from 
%
\begin{lstlisting}https://github.com/spdanda/AI1103/edit/main/Challenging%20Problem1/Result_from_ChallengingProb1/main.tex
\end{lstlisting}
\large\textbf{Statement :}\\
A non-negative random variable has zero expectation value if and only if it is zero.\\
\textbf{Proof :}\\
Let $Y$ be a non-negative random variable defined on the probability space $\Omega$.\\
Let $A_m$ be the set of all $\omega \in \Omega$ such that the value of $Y$ is greater than $\frac{1}{m}$ where $m \in N$ i.e.,
$$A_m = \cbrak{\omega \in \Omega\,:\,Y(\omega) \geq \frac{1}{m}}$$
Given
\begin{align}
    E(Y) =0 \implies \int_\Omega Yd_P =0
\end{align}
As $Y$ is non-negative,
\begin{align}
    \int_\Omega Yd_P \geq \int_{A_m}Yd_p \label{simul1}
\end{align}
Also in the set $A_m$, $Y \geq \frac{1}{m}$
\begin{align}
  \therefore \int_{A_m}Yd_p  &\geq  \int_{A_m}\frac{1}{m}d_p\\
                             &= \frac{1}{m}\int_{A_m}d_p\\
                             &= \frac{1}{m}\,\pr{A_m} \label{simul2}
\end{align}
$\therefore$ From \eqref{simul1} and \eqref{simul2}
\begin{align}
  \int_\Omega Yd_P \geq \frac{1}{m}\,\pr{A_m}
\end{align}
\begin{align}
    \implies \frac{1}{m}\,\pr{A_m} &\leq 0\\
    \implies \pr{A_m} &\leq 0\;(\because m \in N)\\
    \implies \pr{A_m} &= 0
\end{align}
Also,
\begin{multline}
    \pr{Y \neq 0} = \pr{\bigcup A_m}\\(\because \text{In the set } A_m,\,Y \geq \frac{1}{m} \implies Y\,\text{always}> \,0)
\end{multline}
\begin{align}
    &= \pr{A_1 + A_2 + A_3 + ....}\\
    &= 0\,(\because \pr{A_m} = 0\;\forall m \in N)
\end{align}
\begin{align}
  \implies \pr{Y \neq 0} &= 0\\
  \implies \pr{Y =0} &= 1- \pr{Y \neq 0} = 1
\end{align}
Hence, Probability of the random variable equals to 0 is one i.e., the random variable is always equals to 0 in it's domain.\\\\
Conversely, If Y=0 then
\begin{align}
    \int_\Omega Yd_P &=0\\ 
    \implies E(Y) &=0 
\end{align}

Hence proved.






\end{document}
