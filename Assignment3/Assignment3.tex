\documentclass[journal,12pt,twocolumn]{IEEEtran}

\usepackage{setspace}
\usepackage{gensymb}
\singlespacing
\usepackage[cmex10]{amsmath}
\usepackage{amsmath}

\usepackage{amsthm}

\usepackage{mathrsfs}
\usepackage{txfonts}
\usepackage{stfloats}
\usepackage{bm}
\usepackage{cite}
\usepackage{cases}
\usepackage{subfig}

\usepackage{longtable}
\usepackage{multirow}
\usepackage{caption}

\usepackage{enumitem}
\usepackage{mathtools}
\usepackage{steinmetz}
\usepackage{tikz}
\usepackage{circuitikz}
\usepackage{verbatim}
\usepackage{tfrupee}
\usepackage[breaklinks=true]{hyperref}
\usepackage{graphicx}
\usepackage{tkz-euclide}
\usepackage{float}
\usepackage{mathtools}

\usetikzlibrary{calc,math}
\usepackage{listings}
    \usepackage{color}                                            %%
    \usepackage{array}                                            %%
    \usepackage{longtable}                                        %%
    \usepackage{calc}                                             %%
    \usepackage{multirow}                                         %%
    \usepackage{hhline}                                           %%
    \usepackage{ifthen}                                           %%
    \usepackage{lscape}     
\usepackage{multicol}
\usepackage{chngcntr}

\DeclareMathOperator*{\Res}{Res}
\newcommand*{\permcomb}[4][0mu]{{{}^{#3}\mkern#1#2_{#4}}}
\newcommand*{\perm}[1][-3mu]{\permcomb[#1]{P}}
\newcommand*{\comb}[1][-1mu]{\permcomb[#1]{C}}
\renewcommand\thesection{\arabic{section}}
\renewcommand\thesubsection{\thesection.\arabic{subsection}}
\renewcommand\thesubsubsection{\thesubsection.\arabic{subsubsection}}

\renewcommand\thesectiondis{\arabic{section}}
\renewcommand\thesubsectiondis{\thesectiondis.\arabic{subsection}}
\renewcommand\thesubsubsectiondis{\thesubsectiondis.\arabic{subsubsection}}


\hyphenation{op-tical net-works semi-conduc-tor}
\def\inputGnumericTable{}                                 %%

\lstset{
%language=C,
frame=single, 
breaklines=true,
columns=fullflexible
}
\begin{document}

\newcommand{\BEQA}{\begin{eqnarray}}
\newcommand{\EEQA}{\end{eqnarray}}
\newcommand{\define}{\stackrel{\triangle}{=}}
\bibliographystyle{IEEEtran}
\raggedbottom
\setlength{\parindent}{0pt}
\providecommand{\mbf}{\mathbf}
\providecommand{\pr}[1]{\ensuremath{\Pr\left(#1\right)}}
\providecommand{\qfunc}[1]{\ensuremath{Q\left(#1\right)}}
\providecommand{\sbrak}[1]{\ensuremath{{}\left[#1\right]}}
\providecommand{\lsbrak}[1]{\ensuremath{{}\left[#1\right.}}
\providecommand{\rsbrak}[1]{\ensuremath{{}\left.#1\right]}}
\providecommand{\brak}[1]{\ensuremath{\left(#1\right)}}
\providecommand{\lbrak}[1]{\ensuremath{\left(#1\right.}}
\providecommand{\rbrak}[1]{\ensuremath{\left.#1\right)}}
\providecommand{\cbrak}[1]{\ensuremath{\left\{#1\right\}}}
\providecommand{\lcbrak}[1]{\ensuremath{\left\{#1\right.}}
\providecommand{\rcbrak}[1]{\ensuremath{\left.#1\right\}}}
\theoremstyle{remark}
\newtheorem{rem}{Remark}
\newcommand{\sgn}{\mathop{\mathrm{sgn}}}
\providecommand{\abs}[1]{\vert#1\vert}
\providecommand{\res}[1]{\Res\displaylimits_{#1}} 
\providecommand{\norm}[1]{\lVert#1\rVert}
%\providecommand{\norm}[1]{\lVert#1\rVert}
\providecommand{\mtx}[1]{\mathbf{#1}}
\providecommand{\mean}[1]{E[ #1 ]}
\providecommand{\fourier}{\overset{\mathcal{F}}{ \rightleftharpoons}}
%\providecommand{\hilbert}{\overset{\mathcal{H}}{ \rightleftharpoons}}
\providecommand{\system}{\overset{\mathcal{H}}{ \longleftrightarrow}}
	%\newcommand{\solution}[2]{\textbf{Solution:}{#1}}
\newcommand{\solution}{\noindent \textbf{Solution: }}
\newcommand{\cosec}{\,\text{cosec}\,}
\providecommand{\dec}[2]{\ensuremath{\overset{#1}{\underset{#2}{\gtrless}}}}
\newcommand{\myvec}[1]{\ensuremath{\begin{pmatrix}#1\end{pmatrix}}}
\newcommand{\mydet}[1]{\ensuremath{\begin{vmatrix}#1\end{vmatrix}}}
\numberwithin{equation}{subsection}
\makeatletter
\@addtoreset{figure}{problem}
\makeatother
\let\StandardTheFigure\thefigure
\let\vec\mathbf
\renewcommand{\thefigure}{\theproblem}
\def\putbox#1#2#3{\makebox[0in][l]{\makebox[#1][l]{}\raisebox{\baselineskip}[0in][0in]{\raisebox{#2}[0in][0in]{#3}}}}
     \def\rightbox#1{\makebox[0in][r]{#1}}
     \def\centbox#1{\makebox[0in]{#1}}
     \def\topbox#1{\raisebox{-\baselineskip}[0in][0in]{#1}}
     \def\midbox#1{\raisebox{-0.5\baselineskip}[0in][0in]{#1}}
\vspace{3cm}
\title{Assignment-3}
\author{Name: Sai Pravallika Danda, Roll Number: CS20BTECH11013}
\maketitle
\newpage
\bigskip
\renewcommand{\thefigure}{\theenumi}
\renewcommand{\thetable}{\theenumi}
Download all latex-tikz codes from 
%
\begin{lstlisting}
https://github.com/spdanda/AI1103/blob/main/Assignment2/Assignment2.tex
\end{lstlisting}
\large\textbf{UGC mathA-Dec2017 Q59 :}\\
Let $X$ and $Y$ be independent random variables. If $E[X]=1$ and $E[Y]=\frac{1}{2}$ then $\pr{X>2Y|X>Y}$ is
\begin{multicols}{2}
    \begin{enumerate}[label=\arabic*.]
        \item \Large$\frac{1}{2}$ \\
        \item $\frac{1}{3}$
        \item $\frac{2}{3}$ \\
        \item $\frac{3}{4}$
    \end{enumerate}
\end{multicols}
\textbf{Solution :}\\
Since $X$ and $Y$ are exponential random variables with means
\begin{align}
    E[X] = 1 \text{ and }
    E[Y] = \frac{1}{2}
\end{align}
Marginal PDFs of X and Y are given by
\begin{align}
    f_X(x)= e^{-x} , x>0 \\
    f_Y(y) = 2e^{-2y} , y>0
\end{align}
Since $X$ and $Y$ are independent
\begin{align}
    f_{XY}(x,y) &= f_X(x)\times f_Y(y) \;\;\;x,y >0 \\
                &= 2e^{-x}e^{-2y}
\end{align}
Now,
\begin{align}
    \pr{X>2Y|X>Y} &= \dfrac{\pr{(X>2Y) \cap (X>Y)} }{\pr{X>Y}}\\
                  &= \dfrac{\pr{X>2Y}}{\pr{X>Y}} \label{simul1}
\end{align}
\begin{align}
    \pr{X>2Y} &= \int_{0}^{\infty}\int_{0}^{\frac{x}{2}} f_{XY}(x,y) d_yd_x \\
              &= \int_{0}^{\infty}\int_{0}^{\frac{x}{2}} 2e^{-x}e^{-2y}d_yd_x \\
              &= 2\int_{0}^{\infty} e^{-x} \left[\frac{e^{-2y}}{-2}\right]_0^{\frac{x}{2}}d_x\\
              &= \int_{0}^{\infty} e^{-x}(1-e^{-x})d_x\\
              &= \left[\frac{e^{-x}}{-1} - \frac{e^{-2x}}{-2}\right]_0^\infty\\
              &= (0+1) + \frac{1}{2}(0-1)\\
              &= \frac{1}{2} \label{simul2}
\end{align}
\begin{align}
    \pr{X>Y} &= \int_{0}^{\infty}\int_{0}^{x} f_{XY}(x,y) d_yd_x \\
             &= \int_{0}^{\infty}\int_{0}^{x}2e^{-x}e^{-2y}d_yd_x \\
             &= 2\int_{0}^{\infty}e^{-x}\left[\frac{e^{-2y}}{-2}\right]_0^{x}d_x\\
             &= \int_{0}^{\infty}e^{-x}(1-e^{-2x})d_x\\
             &= \left[\frac{e^{-x}}{-1} - \frac{e^{-3x}}{-3}\right]_0^\infty\\
             &= \frac{2}{3} \label{simul3}
\end{align}

Putting \eqref{simul2} and \eqref{simul3} in \eqref{simul1}
\begin{align}
    \pr{X>2Y|X>Y} &= \frac{1/2}{2/3}\\
                  &= \frac{3}{4}
\end{align}

$\therefore$ Option4 is the correct answer.

\end{document}
