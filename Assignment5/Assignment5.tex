\documentclass[journal,12pt,twocolumn]{IEEEtran}

\usepackage{setspace}
\usepackage{gensymb}
\singlespacing
\usepackage[cmex10]{amsmath}
\usepackage{amsmath}

\usepackage{amsthm}

\usepackage{mathrsfs}
\usepackage{txfonts}
\usepackage{stfloats}
\usepackage{bm}
\usepackage{cite}
\usepackage{cases}
\usepackage{subfig}

\usepackage{longtable}
\usepackage{multirow}
\usepackage{caption}

\usepackage{enumitem}
\usepackage{mathtools}
\usepackage{steinmetz}
\usepackage{tikz}
\usepackage{circuitikz}
\usepackage{verbatim}
\usepackage{tfrupee}
\usepackage[breaklinks=true]{hyperref}
\usepackage{graphicx}
\usepackage{tkz-euclide}
\usepackage{float}
\usepackage{mathtools}
\usepackage{multicol}



\usetikzlibrary{calc,math}
\usepackage{listings}
    \usepackage{color}                                            %%
    \usepackage{array}                                            %%
    \usepackage{longtable}                                        %%
    \usepackage{calc}                                             %%
    \usepackage{multirow}                                         %%
    \usepackage{hhline}                                           %%
    \usepackage{ifthen}                                           %%
    \usepackage{lscape}     
\usepackage{multicol}
\usepackage{chngcntr}

\DeclareMathOperator*{\Res}{Res}
\newcommand*{\permcomb}[4][0mu]{{{}^{#3}\mkern#1#2_{#4}}}
\newcommand*{\perm}[1][-3mu]{\permcomb[#1]{P}}
\newcommand*{\comb}[1][-1mu]{\permcomb[#1]{C}}
\renewcommand\thesection{\arabic{section}}
\renewcommand\thesubsection{\thesection.\arabic{subsection}}
\renewcommand\thesubsubsection{\thesubsection.\arabic{subsubsection}}

\renewcommand\thesectiondis{\arabic{section}}
\renewcommand\thesubsectiondis{\thesectiondis.\arabic{subsection}}
\renewcommand\thesubsubsectiondis{\thesubsectiondis.\arabic{subsubsection}}
\renewcommand{\thefigure}{\theenumi}
\renewcommand{\thetable}{\theenumi}


\hyphenation{op-tical net-works semi-conduc-tor}
\def\inputGnumericTable{}                                 %%

\lstset{
%language=C,
frame=single, 
breaklines=true,
columns=fullflexible
}
\newenvironment{Proof}[2] {\textbf{Solution : }}{\hfill$\cdot$}
\begin{document}

\newcommand{\BEQA}{\begin{eqnarray}}
\newcommand{\EEQA}{\end{eqnarray}}
\newcommand{\define}{\stackrel{\triangle}{=}}
\bibliographystyle{IEEEtran}
\raggedbottom
\setlength{\parindent}{0pt}
\providecommand{\mbf}{\mathbf}
\providecommand{\pr}[1]{\ensuremath{\Pr\left(#1\right)}}
\providecommand{\qfunc}[1]{\ensuremath{Q\left(#1\right)}}
\providecommand{\sbrak}[1]{\ensuremath{{}\left[#1\right]}}
\providecommand{\lsbrak}[1]{\ensuremath{{}\left[#1\right.}}
\providecommand{\rsbrak}[1]{\ensuremath{{}\left.#1\right]}}
\providecommand{\brak}[1]{\ensuremath{\left(#1\right)}}
\providecommand{\lbrak}[1]{\ensuremath{\left(#1\right.}}
\providecommand{\rbrak}[1]{\ensuremath{\left.#1\right)}}
\providecommand{\cbrak}[1]{\ensuremath{\left\{#1\right\}}}
\providecommand{\lcbrak}[1]{\ensuremath{\left\{#1\right.}}
\providecommand{\rcbrak}[1]{\ensuremath{\left.#1\right\}}}
\newtheorem{theorem}{Theorem}[section]
\newtheorem{lemma}[theorem]{Lemma}

\theoremstyle{definition}
\newtheorem{definition}{Definition}[section]

\newcommand{\sgn}{\mathop{\mathrm{sgn}}}
\providecommand{\abs}[1]{\vert#1\vert}
\providecommand{\res}[1]{\Res\displaylimits_{#1}} 
\providecommand{\norm}[1]{\lVert#1\rVert}
%\providecommand{\norm}[1]{\lVert#1\rVert}
\providecommand{\mtx}[1]{\mathbf{#1}}
\providecommand{\mean}[1]{E[ #1 ]}
\providecommand{\fourier}{\overset{\mathcal{F}}{ \rightleftharpoons}}
%\providecommand{\hilbert}{\overset{\mathcal{H}}{ \rightleftharpoons}}
\providecommand{\system}{\overset{\mathcal{H}}{ \longleftrightarrow}}
	%\newcommand{\solution}[2]{\textbf{Solution:}{#1}}
\newcommand{\solution}{\noindent \textbf{Solution: }}
\newcommand{\cosec}{\,\text{cosec}\,}
\providecommand{\dec}[2]{\ensuremath{\overset{#1}{\underset{#2}{\gtrless}}}}
\newcommand{\myvec}[1]{\ensuremath{\begin{pmatrix}#1\end{pmatrix}}}
\newcommand{\mydet}[1]{\ensuremath{\begin{vmatrix}#1\end{vmatrix}}}
\numberwithin{equation}{subsection}
\makeatletter
\@addtoreset{figure}{problem}
\makeatother
\let\StandardTheFigure\thefigure
\let\vec\mathbf
\renewcommand{\thefigure}{\theproblem}
\def\putbox#1#2#3{\makebox[0in][l]{\makebox[#1][l]{}\raisebox{\baselineskip}[0in][0in]{\raisebox{#2}[0in][0in]{#3}}}}
     \def\rightbox#1{\makebox[0in][r]{#1}}
     \def\centbox#1{\makebox[0in]{#1}}
     \def\topbox#1{\raisebox{-\baselineskip}[0in][0in]{#1}}
     \def\midbox#1{\raisebox{-0.5\baselineskip}[0in][0in]{#1}}
\vspace{3cm}
\title{Assignment-5}
\author{Name: Sai Pravallika Danda, Roll Number: CS20BTECH11013}
\maketitle
\newpage
\bigskip
\renewcommand{\thefigure}{\theenumi}
\renewcommand{\thetable}{\theenumi}
Download all latex-tikz codes from
\begin{lstlisting}
   https://github.com/spdanda/AI1103/blob/main/Assignment5/Assignment5.tex
\end{lstlisting}
\large\textbf{CSIR-UGC-NET June-2016 Q50\,:}\\
 Let X and Y be independent and identically distributed random variables such that \(\pr{X=0}=\pr{X=1}= \frac{1}{2}\). Let \(Z = X+Y \text{ and } W = \abs{X-Y}\). Then which statement is not correct?
 \begin{enumerate}
     \item $X \text{ and } W$ are independent.
     \item $Y \text{ and } W$ are independent.
     \item $Z \text{ and } W$ are uncorrelated.
     \item $Z \text{ and } W$ are independent.
 \end{enumerate}
\textbf{Solution :}\\
$X, Y \in \cbrak{0,1}
\implies Z \in \cbrak{0,1,2} \text{ and } W \in \cbrak{0,1}$.

\begin{definition}
PMF's for the given random variables $X$ and $Y$ are
 \begin{equation}
         \pr{X=x} = 
     \begin{cases}
      \frac{1}{2} & x \in \cbrak{0,1}\\
      0 & \text{otherwise}
     \end{cases}
 \end{equation}
  \begin{equation}
         \pr{Y=y} = 
     \begin{cases}
      \frac{1}{2} & y \in \cbrak{0,1}\\
      0 & \text{otherwise}
     \end{cases}
 \end{equation}
\end{definition}

\begin{definition}
Probability Generating functions(PGF's) for the random variables $X$ and $Y$ are
\begin{align}
    \mathcal{G}_{X}(z) &= E[z^x]\\
                       &= \sum_{i=0}^{1}p_i\,z^i\\
                       &=\frac{1+z}{2}
\end{align}
Similarly,
\begin{align}
    \mathcal{G}_{Y}(z) = \frac{1+z}{2}
\end{align}
\end{definition}

\begin{lemma}
Generating function for $W = \abs{X-Y}$ where $\mathcal{G}_{X}(z) =\dfrac{1+z}{2}$ and $\mathcal{G}_{Y}(z) = \dfrac{1+z}{2}$ is ($X \text{ and } Y$ being independent) $$\mathcal{G}_W(z) = \frac{1+z}{2}$$
\end{lemma}
\begin{proof}
 \begin{align}
    \mathcal{G}_W(z) &= E\left[z^{\abs{X-Y\,}}\right]
\end{align}
\begin{multline}
\implies  \mathcal{G}_W(z) = E\left[z^{X-Y}\,|(X>Y)\right] +\\ E\left[z^{Y-X}\,|(Y>X)\right] + E\left[z^{X-Y}\,|(X=Y)\right]
\end{multline}
\begin{multline}
\implies  \mathcal{G}_W(z) = \sum\pr{X,Y\,| X>Y}z^{X-Y} + \\\sum\pr{X,Y\,| X<Y}z^{-(X-Y)} + \\\sum\pr{X,Y\,| X=Y} z^{X-Y}
\end{multline}
\begin{center}
\begin{table}[h]
    \centering
    \resizebox{\columnwidth}{!}{
\begin{tabular}{|c|c|c|}
\hline
Case & Possibilities for $(X,Y)$  & $\pr{X,Y}$ \\
\hline
$X>Y$ & (1,0) & $\frac{1}{2}\times\frac{1}{2} = \frac{1}{4}$ \\ 
\hline
$X<Y$ & (0,1) & $\frac{1}{2}\times\frac{1}{2} = \frac{1}{4}$ \\ 
\hline
\multirow{2}{3em}{$X=Y$} & (0,0) & $\frac{1}{2}\times\frac{1}{2} = \frac{1}{4}$ \\ \cline{2-3}
& (1,1) & $\frac{1}{2}\times\frac{1}{2} = \frac{1}{4}$ \\ 
\hline
\end{tabular}
}
    \caption{\large Probability Table for $(X,Y)$ in different cases.}
    \label{Table 1}
\end{table}
\end{center}
$\therefore$From the table,
\begin{multline}
    \mathcal{G}_W(z) = [\pr{X,Y}z^{X-Y}]_{(X,Y)=(1,0)} +\\ [\pr{X,Y}z^{Y-X}]_{(X,Y)=(0,1)} +\\ [\pr{X,Y}z^{X-Y}]_{(X,Y)=(0,0)} +\\ [\pr{X,Y}z^{X-Y}]_{(X,Y)=(1,1)}
\end{multline}
\begin{align}
    &= \frac{1}{4}z^{(1-0)} + \frac{1}{4}z^{(1-0)}+\left(\frac{1}{4}z^{(0-0)} + \frac{1}{4}z^{(1-1)}\right)\\
   &= \dfrac{z}{2} + \dfrac{1}{2}\\
   &= \dfrac{1+z}{2}
\end{align}
\end{proof}

\begin{lemma}
Expected value of $W$ with $\mathcal{G}_W(z) = \dfrac{1+z}{2}$ is $\frac{1}{2}$
\end{lemma}
\begin{proof}
 As $\mathcal{G}_W(z) = \dfrac{1+z}{2}$, pmf of $W$ is 
 \begin{equation}
   \pr{W=w} =
    \begin{cases}
      \frac{1}{2} & w \in \cbrak{0,1}\\
      0 & \text{otherwise}
    \end{cases}       
\end{equation}
So,
\begin{align}
   E[W] &= 0\times\frac{1}{2} + 1\times\frac{1}{2} \\
        &= \frac{1}{2}
\end{align}
\end{proof}

\begin{lemma}
Generating Function for $Z=X+Y$ is ($X \text{ and } Y$ being independent) $$\mathcal{G}_Z(z)=\mathcal{G}_X(z) \times \mathcal{G}_Y(z)$$ 
\end{lemma}
\begin{proof}
 \begin{align}
  \mathcal{G}_Z(z) &= E[z^{X+Y}]\\
                   &= E[z^X z^Y]\\
                   &= E[z^X]\times E[z^Y]
\end{align}
($\because X \text{ and } Y$ are independent)
\begin{align}
    &= \mathcal{G}_X(z) \times \mathcal{G}_Y(z)
\end{align}
\end{proof}
 $\therefore$ From \textbf{Lemma 0.2};
\begin{align}
\mathcal{G}_Z(z) &=\left(\frac{1 + z}{2}\right)^2\\
                   &= \frac{1+2z+z^2}{4}\\
\end{align}
\begin{lemma}
Expected value of $Z$ with $\mathcal{G}_Z(z) = \dfrac{1+2z+z^2}{4}$ is 1 
\end{lemma}
\begin{proof}
 As
 \begin{align}
     \mathcal{G}_Z(z) &= \frac{1+2z+z^2}{4}\\
                      &= \frac{1}{4} + \frac{1}{2}z + \frac{1}{4}z^2
 \end{align}
 PMF of $Z$ is
 \begin{align}
    \pr{Z = z} = 
\begin{cases}
\frac{1}{4} & z=0
\\
\frac{1}{2} & z=1\\
\frac{1}{4} & z=2
\end{cases}\label{2}
\end{align}
$\therefore$ Expected value of $Z$ is
\begin{align}
   E[Z] &= 0\times\frac{1}{4} + 1\times\frac{1}{2} + 2\times\frac{1}{4}\\
                     &= 1
\end{align}
\end{proof}

Now, checking for each option,
\begin{enumerate}
    \item Checking if $X$ and $W$ are independent:
\begin{align}
    p_1 &= \pr{X=x,W=w}\\
        &= \pr{X=x,Y=x\pm w}\\
        &=\pr{X=x}\times\pr{Y=x\pm w}
\end{align}
($\because\,X$ and $Y$ are independent)\\\\ 
Note that here $Y \in \cbrak{0,1}$.\\
So Possible values of $Y$ for different values of $x$ and $w$ are
\begin{center}
\begin{table}[h]
    \centering
    \resizebox{\columnwidth}{!}{
\begin{tabular}{|c|c|c|c|}
\hline
$x$ & $w$ & Possibilities for $Y$ & $\pr{Y}$ \\
\hline
0 & 0 & 0 & $\frac{1}{2}$\\ 
\hline
0 & 1 & 1 & $\frac{1}{2}$ \\ 
\hline
1 & 0 & 1 & $\frac{1}{2}$\\
\hline
1 & 1 & 0 & $\frac{1}{2}$\\
\hline
\end{tabular}
}
    \caption{\large Probability Table for $Y$ when different $x$ and $w$'s are substituted}
    \label{Table 2}
\end{table}
\end{center}
So, the value of $\pr{Y}\;\forall$ values of $x,\,w$ is equals to $\frac{1}{2}$\\ 
$\implies\,\pr{X=x}\times\pr{Y=x\pm w}$ $\forall$ values of $x,w\in \cbrak{0,1}$ is equals to $\frac{1}{2}\times\frac{1}{2}$
\begin{align}
\implies p_1 &= \frac{1}{2}\times\frac{1}{2}\\
             &= \frac{1}{4}
\end{align}
\begin{align}
   \therefore \pr{X=x,W=w} = \frac{1}{4}\;\forall x,w\in \cbrak{0,1} \label{simul2}
\end{align}
Also,
\begin{align}
\nonumber\pr{X=x}\times\pr{W=w} &= \frac{1}{2} \times\frac{1}{2}\\
                                &\;\forall x,w\in \cbrak{0,1}  \\ 
                                &= \frac{1}{4} \label{simul3}
\end{align}
$\therefore$ From \eqref{simul2} and \eqref{simul3}.
\begin{multline}
    \Pr{(X=x)}\Pr{(W=w)} =\\ \Pr{(X=x,W=w)}
\end{multline}

$\implies$ $X$ and $W$ are independent and hence Option 1 is true.
\item Checking if $Y$ and $W$ are independent :\\
Solving this case is identical to the first option except the variable $X$ is replaced by $Y$ (Note that here $W$ is symmetric wrt to $X$ and $Y$).\\
Hence on solving, you get $Y$  and $W$ are independent.\\
$\therefore$ Option 2 is also true.\\
4) Checking if $W$ and $Z$ are independent:\\
Let's check for one particular case.
\begin{align}
    \pr{W=0\,|\,Z=0} = 1
\end{align}
($\because\;Z=0 \implies X = Y =0 \implies W=0$)
\begin{align}
    \therefore \frac{\pr{W=0,\,Z=0}}{\pr{Z=0}} &= 1
\end{align}
\begin{align}
 \implies\pr{W=0,Z=0} &=\pr{Z=0}\\ 
                      &= \frac{1}{4}
\end{align}
\begin{align}
\hspace{2cm}&\neq\pr{W=0}\times\pr{Z=0}   
\end{align}
$\therefore$ $W$ and $Z$ are not independent random variables\\
Hence option 4 is false.
\item Checking if $W$ and $Z$ are uncorrelated:\\
\textbf{Uncorrelated random variables:} Two variables are said to be uncorrelated if the expected value of their joint distribution is equal to product of the expected values of their respective marginal distributions.\\
Let $U=WZ\;\implies U\in\cbrak{0,1,2}\,(\because W \in\cbrak{0,1} \text{ and } Z \in\cbrak{0,1,2})$
\begin{lemma}
PMF for the random variable $U$ is
 \begin{align}
    \pr{U = u} = 
\begin{cases}
\frac{1}{2} & u=0
\\
\frac{1}{2} & u=1\\
0 & u=2
\end{cases}
\end{align}
\end{lemma}
\begin{proof}
\begin{align}
    \pr{U=2} &=\pr{WZ =2} \\
             &= \pr{W=1,Z=2} =0
\end{align}
($\because$ If $Z=2$, $X = Y =1 \implies W =0$ i.e., $\neq$ to 1)
$$\therefore\,\pr{U=2}=0$$
\begin{align}
   \pr{U=0} &=\pr{WZ =0}
\end{align}
So, either $W=0\;\text{or}\;Z =0$\\
$\implies X=Y=0 \text{ or } X=Y=1$
\begin{multline}
       \pr{U=0} = \pr{X=0,Y=0} + \\ \pr{X=1,Y=1} 
\end{multline}
\begin{multline}
\hspace{1cm}     = \pr{X=0}\pr{Y=0} +\\ \pr{X=1}\pr{Y=1}
\end{multline}
\begin{align}
    &= \frac{1}{4} + \frac{1}{4}\\
    &= \frac{1}{2}
\end{align}
$$\therefore\,\pr{U=0} = \frac{1}{2}$$
Now,
\begin{multline}
    \pr{U=1}= 1 -\\\left[\pr{U =0} -\pr{U =2}\right]
\end{multline}
\begin{align}
    &= 1 -\left[\frac{1}{2} + 0\right]\\
    &= \frac{1}{2}
\end{align}
$$\therefore\,\pr{U=1} = \frac{1}{2}$$
Hence PMF of $U$ is
\begin{align}
    \pr{U = u} = 
\begin{cases}
\frac{1}{2} & u=0
\\
\frac{1}{2} & u=1\\
0 & u=2
\end{cases}
\end{align}
\end{proof}
Expected value of $WZ$, $E[WZ]$
\begin{align}
    &= \sum_{k=0,1,2} k\pr{WZ =k}\\
    &= 1 \times \frac{1}{2} +0\\
    &= \frac{1}{2} 
\end{align}
\begin{align}
    \text{Also } E[W]\times E[Z] &= \frac{1}{2}\times1 = \frac{1}{2}\\
                                 &= E[WZ] \label{simul4}
\end{align}
Hence from \eqref{simul4}\,$W$ and $Z$ are uncorrelated random variables.\\
$\therefore$Option 3 is also true.\\

$\therefore$ Incorrect Option is 4.
    
\end{enumerate}
 
\end{document}
