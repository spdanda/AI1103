\documentclass[journal,12pt,twocolumn]{IEEEtran}

\usepackage{setspace}
\usepackage{gensymb}
\singlespacing
\usepackage[cmex10]{amsmath}
\usepackage{amsmath}

\usepackage{amsthm}

\usepackage{mathrsfs}
\usepackage{txfonts}
\usepackage{stfloats}
\usepackage{bm}
\usepackage{cite}
\usepackage{cases}
\usepackage{subfig}

\usepackage{longtable}
\usepackage{multirow}
\usepackage{caption}

\usepackage{enumitem}
\usepackage{mathtools}
\usepackage{steinmetz}
\usepackage{tikz}
\usepackage{circuitikz}
\usepackage{verbatim}
\usepackage{tfrupee}
\usepackage[breaklinks=true]{hyperref}
\usepackage{graphicx}
\usepackage{tkz-euclide}
\usepackage{float}
\usepackage{mathtools}
\usepackage{multicol}



\usetikzlibrary{calc,math}
\usepackage{listings}
    \usepackage{color}                                            %%
    \usepackage{array}                                            %%
    \usepackage{longtable}                                        %%
    \usepackage{calc}                                             %%
    \usepackage{multirow}                                         %%
    \usepackage{hhline}                                           %%
    \usepackage{ifthen}                                           %%
    \usepackage{lscape}     
\usepackage{multicol}
\usepackage{chngcntr}

\DeclareMathOperator*{\Res}{Res}
\newcommand*{\permcomb}[4][0mu]{{{}^{#3}\mkern#1#2_{#4}}}
\newcommand*{\perm}[1][-3mu]{\permcomb[#1]{P}}
\newcommand*{\comb}[1][-1mu]{\permcomb[#1]{C}}
\renewcommand\thesection{\arabic{section}}
\renewcommand\thesubsection{\thesection.\arabic{subsection}}
\renewcommand\thesubsubsection{\thesubsection.\arabic{subsubsection}}

\renewcommand\thesectiondis{\arabic{section}}
\renewcommand\thesubsectiondis{\thesectiondis.\arabic{subsection}}
\renewcommand\thesubsubsectiondis{\thesubsectiondis.\arabic{subsubsection}}


\hyphenation{op-tical net-works semi-conduc-tor}
\def\inputGnumericTable{}                                 %%

\lstset{
%language=C,
frame=single, 
breaklines=true,
columns=fullflexible
}
\begin{document}

\newcommand{\BEQA}{\begin{eqnarray}}
\newcommand{\EEQA}{\end{eqnarray}}
\newcommand{\define}{\stackrel{\triangle}{=}}
\bibliographystyle{IEEEtran}
\raggedbottom
\setlength{\parindent}{0pt}
\providecommand{\mbf}{\mathbf}
\providecommand{\pr}[1]{\ensuremath{\Pr\left(#1\right)}}
\providecommand{\qfunc}[1]{\ensuremath{Q\left(#1\right)}}
\providecommand{\sbrak}[1]{\ensuremath{{}\left[#1\right]}}
\providecommand{\lsbrak}[1]{\ensuremath{{}\left[#1\right.}}
\providecommand{\rsbrak}[1]{\ensuremath{{}\left.#1\right]}}
\providecommand{\brak}[1]{\ensuremath{\left(#1\right)}}
\providecommand{\lbrak}[1]{\ensuremath{\left(#1\right.}}
\providecommand{\rbrak}[1]{\ensuremath{\left.#1\right)}}
\providecommand{\cbrak}[1]{\ensuremath{\left\{#1\right\}}}
\providecommand{\lcbrak}[1]{\ensuremath{\left\{#1\right.}}
\providecommand{\rcbrak}[1]{\ensuremath{\left.#1\right\}}}
\theoremstyle{remark}
\newtheorem{rem}{Remark}
\newcommand{\sgn}{\mathop{\mathrm{sgn}}}
\providecommand{\abs}[1]{\vert#1\vert}
\providecommand{\res}[1]{\Res\displaylimits_{#1}} 
\providecommand{\norm}[1]{\lVert#1\rVert}
%\providecommand{\norm}[1]{\lVert#1\rVert}
\providecommand{\mtx}[1]{\mathbf{#1}}
\providecommand{\mean}[1]{E[ #1 ]}
\providecommand{\fourier}{\overset{\mathcal{F}}{ \rightleftharpoons}}
%\providecommand{\hilbert}{\overset{\mathcal{H}}{ \rightleftharpoons}}
\providecommand{\system}{\overset{\mathcal{H}}{ \longleftrightarrow}}
	%\newcommand{\solution}[2]{\textbf{Solution:}{#1}}
\newcommand{\solution}{\noindent \textbf{Solution: }}
\newcommand{\cosec}{\,\text{cosec}\,}
\providecommand{\dec}[2]{\ensuremath{\overset{#1}{\underset{#2}{\gtrless}}}}
\newcommand{\myvec}[1]{\ensuremath{\begin{pmatrix}#1\end{pmatrix}}}
\newcommand{\mydet}[1]{\ensuremath{\begin{vmatrix}#1\end{vmatrix}}}
\numberwithin{equation}{subsection}
\makeatletter
\@addtoreset{figure}{problem}
\makeatother
\let\StandardTheFigure\thefigure
\let\vec\mathbf
\renewcommand{\thefigure}{\theproblem}
\def\putbox#1#2#3{\makebox[0in][l]{\makebox[#1][l]{}\raisebox{\baselineskip}[0in][0in]{\raisebox{#2}[0in][0in]{#3}}}}
     \def\rightbox#1{\makebox[0in][r]{#1}}
     \def\centbox#1{\makebox[0in]{#1}}
     \def\topbox#1{\raisebox{-\baselineskip}[0in][0in]{#1}}
     \def\midbox#1{\raisebox{-0.5\baselineskip}[0in][0in]{#1}}
\vspace{3cm}
\title{Assignment-5}
\author{Name: Sai Pravallika Danda, Roll Number: CS20BTECH11013}
\maketitle
\newpage
\bigskip
\renewcommand{\thefigure}{\theenumi}
\renewcommand{\thetable}{\theenumi}
Download all latex-tikz codes from
\begin{lstlisting}
   https://github.com/spdanda/AI1103/blob/main/Assignment5/Assignment5.tex
\end{lstlisting}
\large\textbf{CSIR-UGC-NET June-2016 Q50\,:}\\
 Let X and Y be independent and identically distributed random variables such that \(\pr{X=0}=\pr{X=1}= \frac{1}{2}\). Let \(Z = X+Y \text{ and } W = \abs{X-Y}\). Then which statement is not correct?
 \begin{enumerate}
     \item $X \text{ and } W$ are independent.
     \item $Y \text{ and } W$ are independent.
     \item $Z \text{ and } W$ are uncorrelated.
     \item $Z \text{ and } W$ are independent.
 \end{enumerate}
 \textbf{Solution:}\\
 $X, Y \in \cbrak{0,1}\implies Z \in \cbrak{0,1,2} \text{ and } W \in \cbrak{0,1}$.\\
 Also,
 \begin{equation}
         \pr{X=x} = 
     \begin{cases}
      \frac{1}{2} & x \in \cbrak{0,1}\\
      0 & \text{otherwise}
     \end{cases}
 \end{equation}
  \begin{equation}
         \pr{Y=y} = 
     \begin{cases}
      \frac{1}{2} & y \in \cbrak{0,1}\\
      0 & \text{otherwise}
     \end{cases}
 \end{equation}
 \begin{enumerate}[label=\alph*)]
\item
\begin{multline}
        \pr{W=0} = \pr{X=0,Y=0}+\\\pr{X=1,Y=1}
\end{multline}
\begin{align}
        &= \frac{1}{2}\times\frac{1}{2} + \frac{1}{2}\times\frac{1}{2}\\
        &= \frac{1}{2}
 \end{align}
 \begin{align}
     \pr{W=1} &= 1- \pr{W=0}\\
              &= \frac{1}{2}
 \end{align}
 \begin{equation}
  \therefore \pr{W=w} =
    \begin{cases}
      \frac{1}{2} & w \in \cbrak{0,1}\\
      0 & \text{otherwise}
    \end{cases}       
\end{equation}
\begin{align}
   \text{Also }E[W] &= 0\times\frac{1}{2} + 1\times\frac{1}{2}\\
                    &= \frac{1}{2} 
\end{align}
 \item
 \begin{multline}
     \pr{Z=z} =\pr{X+Y =z}\\
              =\sum_{x=0}^z\pr{X=x}\pr{Y=z-x}
 \end{multline}
 
\begin{align}
          &=\left(2-\abs{z-1}\,\right)\times\frac{1}{2}\times\frac{1}{2}\\
          &= \frac{2-\abs{z-1}}{4}
\end{align}
\begin{equation}
  \therefore \pr{Z=z} =
    \begin{cases}
       \frac{2-\abs{\,z-1}}{4} & z \in \cbrak{0,1,2}\\
      0 & \text{otherwise}
    \end{cases}       
\end{equation}
\begin{align}
    \text{And } E[Z] &= 0\times\frac{1}{4} + 1\times\frac{1}{2} + 2\times\frac{1}{4}\\
                     &= 1
\end{align}
\end{enumerate}
Now, for checking each option,
\begin{enumerate}
    \item Checking if $X$ and $W$ are independent
\begin{align}
    p_1 &= \pr{X=x,W=w}\\
        &= \pr{X=x,Y=x\pm w}\\
        &=\pr{X=x}\times\pr{Y=x\pm w}\\
        &= \frac{1}{2}\times\frac{1}{2}\\
        &= \begin{cases}
        \frac{1}{4} & (x\pm w) \in \{1, 2, 3, 4, 5, 6\}\\ ~\\[-1em]
        0 & \text{otherwise}
    \end{cases}
\end{align}
(only one value for $Y$ is obtained for each case when $x$ and $w$ are substituted)
\begin{align}
    \pr{X=x}\times\pr{W=w} &= \frac{1}{2} \times\frac{1}{2} \\
                           &= \frac{1}{4}
\end{align}
\begin{multline}
    \Pr{(X=x)}\Pr{(W=w)} =\\ \Pr{(X=x,W=w)}
\end{multline}

$\implies$ $X$ and $W$ are independent and hence Option 1 is true.
\item Checking if $Y$ and $W$ are independent\\
Solving of this case is identical to the first option except the variable $X$ is replaced by $Y$.
Hence on solving, you get $Y$  and $W$ are independent.\\
$\therefore$ Option 2 is also true.
\item Checking if $W$ and $Z$ are uncorrelated\\
\textbf{Uncorrelated random variables:} Two variables are said to be uncorrelated if the expected value of their joint distribution is equal to product of the expected values of their respective marginal distributions.\\
Also, $WZ \in\cbrak{0,1,2}$
\begin{align}
    \pr{WZ =2} = \pr{W=1,Z=2} =0
\end{align}
($\because$ If $Z=2$, $X = Y =1 \implies W =0$ i.e., $\neq$ to 1)
\begin{align}
    \pr{WZ =0} = \frac{1}{2}
\end{align}
($\because\text{either }W=0\;\text{or}\;Z =0)$
\begin{multline}
    \pr{WZ=1} = 1 - \pr{WZ =0} - \\\pr{WZ =2}
\end{multline}
\begin{align}
    &= 1 -\frac{1}{2}\\
    &= \frac{1}{2}
\end{align}
$\therefore$ Expected value of $WZ$, $E[WZ]$
\begin{align}
    &= \sum_{k=0,1,2} k\pr{WZ =k}\\
    &= 1 \times \frac{1}{2} +0\\
    &= \frac{1}{2}
\end{align}
\begin{align}
    \text{Also } E[W]\times E[Z] &= \frac{1}{2}\times1 = \frac{1}{2}\\
                                 &= E[WZ]
\end{align}
Hence from the above equation $W$ and $Z$ are uncorrelated random variables.\\
$\therefore$Option 3 is also true.
\item Let's check for one particular case.
\begin{align}
    \pr{W=0\,|\,Z=0} = 1
\end{align}
($\because\;Z=0 \implies X = Y =0 \implies W=0$)
\begin{align}
    \therefore \frac{\pr{W=0,\,Z=0}}{\pr{Z=0}} &= 1
\end{align}
\begin{multline}
   \implies\pr{W=0,Z=0} =\pr{Z=0} \\
                       \neq\pr{W=0}\times\pr{Z=0}   
\end{multline}
$\therefore$ $W$ and $Z$ are not independent random variables\\
Hence Option 4 is incorrect.\\
$\therefore$ Answer is Option4.
    
\end{enumerate}
  
 
 
 
\end{document}
